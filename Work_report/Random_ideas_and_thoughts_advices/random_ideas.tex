\documentclass{article}
\usepackage[a4paper, tmargin=1in, bmargin=1in]{geometry}
\usepackage[utf8]{inputenc}
\usepackage{graphicx}
\usepackage{parskip}
\usepackage{pdflscape}
\usepackage{listings}
\usepackage{hyperref}
% \usepackage{titlesec}

\newcommand{\ra}{$\rightarrow$}


\title{Viterbi Internship - Random Ideas}
\author{
  Arka Sadhu}
\date{\today}

\begin{document}
\maketitle

\tableofcontents
\newpage
\section{Face Detection}
%Date : 17 May 2017, 20:14
\subsection{NN for finding what the hidden layers of the Net are doing}
\begin{itemize}
\item Can try to analyze the hidden layers, to find which nodes are the ones which are capturing illumination invariant features and stuff like that.
\end{itemize}

\section{NN}
\subsection{Filtering automatically with CNN}
\begin{itemize}
\item Can try to remove the data bias from the database, by physically removing them using another NN
\item Or perhaps automatically learning that these are not exactly relevant.
\end{itemize}

\subsection{Training NN}
\begin{itemize}
\item To do a task, we can think of how a normal human being would approach that task.
\item Can ask the machine learner to do that on its own.
\item This would implicitly lead to learning objects on its own.
\end{itemize}

\subsection{Labels rather than Folders}
\begin{itemize}
\item Instead of a dataset which includes only classes, ie image1 either goes to class A or to class B, we can do some slight variations.
  \begin{itemize}
  \item Have tags instead of classes. Each image can be classified into more than one sets. The idea is that the sets are no longer disjoint, rather they have some intersection.
  \item Heirarchial classes. Instead of 1D output neural network, we can have multi dimensional neural network.
  \end{itemize}
\end{itemize}
\subsection{MediFor}
\begin{itemize}
\item One thing that could be done is to divide the problem into two halves.
\item The first is to get the annotations about the image. Something describing some kind of action being performed.
\item Next is to use those annotations to make out if there is anything semantically incorrect.
\end{itemize}
\section{Important Tips : By Prof. Ram Nevatia}
\begin{itemize}
\item Whatever written in the paper, is written very nicely. Do not change the work flow for poster, or even the content or representation. Has happened with the prof said that pains in changing the course plan from one assignment to another. 
\end{itemize}


\end{document}