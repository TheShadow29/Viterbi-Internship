\documentclass{article}
\usepackage[a4paper, tmargin=1in, bmargin=1in]{geometry}
\usepackage[utf8]{inputenc}
\usepackage{graphicx}
\usepackage{parskip}
\usepackage{pdflscape}
\usepackage{listings}
\usepackage{hyperref}
% \usepackage{titlesec}

\newcommand{\ra}{$\rightarrow$}


\title{Viterbi Internship - Progress Diary}
\author{
  Arka Sadhu}
\date{\today}

\begin{document}
\maketitle

\tableofcontents
\newpage
\section{Daily Progress}
\subsection{Week 15 May to 21 May}
\subsubsection{15 May : Monday}
Accomplished :
\begin{itemize}
\item Read the paper Face Recog using deep multi pose representation halfway through.
\item Watched and implemented tensorflow tutorials by Marvon Zhou till Lec15.
\end{itemize}

\subsubsection{16 May : Tuesday}
Target :
\begin{itemize}
\item Complete the Face Recog using deep multi pose representation.
\item Watch NN course  and complete till week 10 (curr status at week 6).
\item Read Do We really need million faces.
\end{itemize}
Accomplished:
\begin{itemize}
\item Completed reading teh Face Recog using deep multi pose representation
\item Started reading Do We really need million faces [upto page 4]
\end{itemize}

\subsubsection{17 May : Wednesday}
Target :
\begin{itemize}
\item Complete Do We really need million faces
\item Complete the other two papers as well : A multi scale cascade fully convolutional network face detector, regressing parameters for 3DMM
\item Finish upto week 10
\end{itemize}
Accomplished :
\begin{itemize}
\item Complted two papers : Do we really need million faces, and multi scale cascade fcn face detector, and started regressing parameters for 3DMM
\end{itemize}

\subsubsection{18 May : Thursday}
Target :
\begin{itemize}
\item Complete regressing 3DMM parameters
\item Complete till week 10 from NN
\item Also try to do the course CS231n Stanford : CNN for Visual Recognition. 
\end{itemize}
Accomplished :
\begin{itemize}
\item Slight part of regressing 3DMM parameters is left, but will leaving it as is.
\item Downloaded the new papers, and had a brief overview regarding that.
\end{itemize}

\subsubsection{19 May : Friday}
Target :
\begin{itemize}
\item Read all the  4 papers regarding MediFor.
\item Complete till week 10 from NN
\item Also see CS231n Stanford course.
\end{itemize}

Accomplished :
\begin{itemize}
\item Got a new Problem Statement :
  Given an image, need to develop a score map which can say wheather or not this image was taken in that location or not.
\item Downloaded places365, need to start working with it.
\end{itemize}

\subsubsection{20 May : Saturday}
No work done.

\subsubsection{21 May : Sunday}
No work done.

\subsection{Week 22 May to 28 May}
\subsubsection{22 May : Monday}
Target:
\begin{itemize}
\item Read im2gps paper, unsupervised visual representation learning by context prediction.
\item Get places365 running nicely, and try to replicate the results.
\item Complete till week 10 of NN course.
\item Understand the problem statement once again.
\item Get a hang of Caffe.
\end{itemize}
Accomplished:
\begin{itemize}
\item Read im2gps paper. Mildly interesting, mostly experiments. Didn't really explain well what it wanted to show.
\item Read the unsupervised Visual representation learning by context prediction.
\item Did the lec6e finally, and started lec7 of NN course.
\end{itemize}

\subsubsection{23 May : Tuesday}
Target:
\begin{itemize}
\item Get hang of how to use Caffe.
\item Get Places-CNN working.
\end{itemize}

Accomplished:
\begin{itemize}
\item Got a good hang of caffe. Tried 3 hands on examples.
\item The places365 website was down for some, reason, couldn't really do anything on that.
\item Read first 3 lectures of CS231n. Tried few hands on examples.
\end{itemize}

\subsubsection{24 May : Wednesday}
Target:
\begin{itemize}
\item Get Places-CNN working on laptop. Train if required. See github for reference (site is also up).
\end{itemize}

Accomplished:
\begin{itemize}
\item Reality is harsher. It takes a lot of time to extract the relevant folders. Damn only if there was any way to make this process a bit more faster.
\item Not feasible to train on the whole Places dataset. Will take enormous amount of time. More beneficial to use that time for testing purposes.
\item Finally set up the PC here for caffe. Took a lot less time than it did on my computer.
\item Tried the flickr\_finetuning tutorial. The results as seen on Thursday were extremely bad. I have posted on caffe-users, but the community is not very responsive for some reason. Not exactly sure where the problem is. But still got a hang of fine tuning at least.
\item Caffe documentation is seriously bad. Need some good tutorials for this. But still got a decent hang of caffe now. Need to start with the medifor dataset at the earliest.
\item Seems like did a lot of not-so-really-useful-things today.
\end{itemize}

\subsubsection{25 May : Thursday}
Target :
\begin{itemize}
\item Try to get Places-CNN testing running some or the other way. Get a hold of the MediFor dataset. It will perhaps take time to preprocess.
\item See CS231n for reference in the mean time, may get some useful ideas.
\end{itemize}

Accomplished :
\begin{itemize}
\item Damn, there seems to be some caution that needs to be taken care for External hard-disk. Note to self : in the future, if you are trying to untar, donot use the direct archive method. The problem isn't exactly keep on storing (writing) the data. In fact, if there is some trouble in the process of untaring, there is practically nothing one can do. I had to remove the Hard-disk from the PC. But then it threw the error that the disk is corrupted. My guess is that, while untaring, it still wanted to write some data, which it was not able to do so. I had to go windows and run chkdsk (probably short for check disk), most likely because it is an ntfs partition. At the very least, chkdsk solved the problem pretty quickly. Now trying to unzip using winRAR, and at the very least it shows ETA. For about 4.4GB test data set of the Places365 dataset, it is taking about 4 hours. Lets see how it goes.
\item I am an idiot. I was trying to untar it on the hard disk. Untaring it on the SDD was like 1min or so. I seriously wasted the whole morning.
\item I am once again reminded of the need to read the documentation. Read the complete caffe documentation, now I feel like I can do something. 
\end{itemize}

\subsubsection{26 May : Friday}
Target : 
\begin{itemize}
\item I think, I should go step by step. First I will try to read the existing caffe model, and try to run that on the places365 dataset. I should try to see the test results, and see if it is meeting the expected benchmarks.
\item Parallely I will try to create some dataset related to the localization. Let's see how it goes from there on.
\end{itemize}

Accomplished :
\begin{itemize}
\item Got places365 to get working. 
\end{itemize}


\section{Weekly Progress}
\subsection{Week 1}
Accomplished:
\begin{itemize}
\item Read 4 papers on Face Recog, but then the project was itself scraped, and I was given a new project.
\item Read 2 papers (2nd paper near to completion) on Friday.
\item Understood the new project on MediFor, but that might be slightly changed in the near future.
\end{itemize}

\subsection{Week 2}
Target :
\begin{itemize}
\item Try to complete the NN course (all weeks[16]).
\item Read papers on the image geolocalization topic.
\item Try to replicate the Places-CNN paper results at the earliest.
\end{itemize}


\end{document}